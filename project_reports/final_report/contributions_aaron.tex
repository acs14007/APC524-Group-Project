%! Author = aaronspaulding
%! Date = 12/16/23


\subsection{An implementation of a Finite Difference Navier-Stokes solver to simulate the two-dimensional cylinder wake flow.}
I implemented a solver for the Navier-Stokes equations using a Finite Difference(FD) method.
This approach allows us to simulate fluid flows in two dimensions quickly and has been applied to many fluid related problems such as weather prediction, aerodynamics, and oceanography.
I implemented this solver in python using the NumPy library for efficient array computations.
The solver initializes the velocity and pressure fields and tracks changes and interactions between ``parcels'' of fluid interacting with each other.
Each parcel is stationary and has a velocity and pressure associated with it that is updated at each time step to track the flux into and out of the parcel.
This solver type enables efficient quick simulations at the cost of using fixed time steps and a fixed grid size.
I implemented this solver following the equations derived by Joseph as well as work published from Barba et al.\cite{barba2018cfd}.


\subsubsection{Environment Setup}
I designed the solver environment to be flexible and modular so users could easily define different size and resolution environments.
The environment has a customizable grid, with adjustable resolution, time step, and fluid properties.
This was implemented as a python ``Environment'' Class inside a module.
The ``Environment'' class also includes automatic plotting routines that enable quick visualization of the fluid flow.

\begin{figure}[!ht]
    \centering
    \begin{subfigure}{.28\textwidth}
        \centering
        \includegraphics[width=\linewidth]{Figures/pipe_flow_example_quiver.png}
        \caption{Simulation of fluid flowing in a pipe. The top and bottom boundary conditions are set as no-slip conditions, while the left and right are set as periodic boundary conditions.}
        \label{fig:sub11}
    \end{subfigure}%
    \hspace{0.04\textwidth}
    \begin{subfigure}{.28\textwidth}
        \centering
        \includegraphics[width=\linewidth]{Figures/two_box_example_streamline.png}
        \caption{Simulation of a fluid flowing around two boxes. Boundary conditions for all sides are set as fixed velocity conditions. The boundary conditions for the boxes are set as no-slip conditions are updated dynamically as each box is added to the environment.}
        \label{fig:sub12}
    \end{subfigure}
    \hspace{0.04\textwidth}
    \begin{subfigure}{.28\textwidth}
        \centering
        \includegraphics[width=\linewidth]{Figures/cylinder_example_streamline.png}
        \caption{A simulation of fluid flow around a cylinder. Here the right, top, and bottom sides have no slip conditions while the left side has a fixed velocity boundary condition. The boundary conditions around the cylinder are automatically updated at simulation time.}
        \label{fig:sub13}
    \end{subfigure}
    \caption{The modular boundary conditions, customizable environment, and composable objects allows for easy simulation of complex environments with very different conditions and requirements.}
    \label{fig:fluid_flow_cylinder_ns_fdm}
\end{figure}


\subsubsection{Boundary Conditions}
To extend this modularity I abstracted different types of boundary conditions common in fluid simulations.
I implemented four different boundary conditions for each edge of the environment.
Each of these boundary conditions is fully modular and can be mixed and matched for each simulation environment.

\begin{enumerate}
    \item \textbf{No-Slip Boundary Condition}: This boundary condition is used to simulate a solid boundary where fluid flows are zero in both the parallel and perpendicular directions to the boundary. This is seen on the inside of pipes, along buildings and objects, and against the ground.
    \item \textbf{Fixed Velocity Boundary Condition}: This boundary condition is used to simulate a boundary where fluid flows are fixed in the parallel direction to the boundary. This could be used to simulate a fan, an inlent valve, or the top of a boundary layer where the fluid is moving at a constant velocity.
    \item \textbf{Periodic Boundary Condition}: This boundary condition is used to simulate a boundary where fluid flows are periodic in the parallel direction to the boundary. This can be used to simulate repeating simulations such as a section of pipe where the input and output ends are similar.
    \item \textbf{Free Slip Boundary Condition}: This boundary condition is used to simulate a boundary where fluid flows are zero in the perpendicular direction to the boundary. This can be used to simulate a boundary where the fluid is free to move in the parallel direction but cannot move in the perpendicular direction.
\end{enumerate}


\subsubsection{Objects in the Environment}
Objects inside environments also interact with fluid flows and affect the velocity and pressure fields.
To enable modular simulations, I also implemented an ``Object'' Class that can be used to place arbitrary objects in the environment.
I implemented a ``Rectangle'' Class that inherits from the abstract ``Object'' Class that automatically manages boundary conditions of the added object and updates the velocity and pressure fields during simulation.
I also implemented a ``Cylinder'' Class that also inherits from the abstract ``Object'' Class.

These can be combined to make complex simulations with multiple objects interacting with each other and the fluid flow.

\subsubsection{Example Simulation Setup}
\begin{lstlisting}[language = Python]
from navier_stokes_fdm import Environment
from navier_stokes_fdm import Rectangle
import navier_stokes_fdm.boundary_condition as bc


U = 1  # m/s
dimension = 0.005

boundary_conditions = [
    bc.TopSideFixedVelocityBoundaryCondition(u_value=U, v_value=0),
    bc.BottomSideFixedVelocityBoundaryCondition(u_value=U, v_value=0),
    bc.LeftSideFixedVelocityBoundaryCondition(u_value=U, v_value=0),
    bc.RightSideFixedVelocityBoundaryCondition(u_value=U, v_value=0),
]


x1, y1 = 0.0125, (0.04 / 2) - (dimension / 2)
objects = [Rectangle(x1, y1, x1 + dimension, y1 + dimension)]


a = Environment(
    F=(1.0, 0.0),
    len_x=0.06,
    len_y=0.04,
    dt=0.00000015,
    dx=0.0001,
    boundary_conditions=boundary_conditions,
    objects=objects,
    rho=0.6125  # kg/m³
    nu=3e-5  # m²/s
)

a.run_many_steps(480)
a.plot_streamline_plot(title="", filepath="../Figures/box_example_streamline.png")
\end{lstlisting}

\begin{figure}[h]
    \centering
    \includegraphics[width=0.5\linewidth]{Figures/box_example_streamline.png}
    \caption{Example streamline plot of a fluid flow around a box. Each boundary is assigned a fixed velocity. Shading represents the pressure field with lighter colors indicating regions of lower pressure. Streamlines are shown in purple.}
    \label{fig:box_example_streamline}
\end{figure}



\subsection{Unit testing}
To further ensure code functionality I implemented unit tests using ``pytest'' for the boundary conditions.
Each test creates an environment, applies a boundary condition, and checks to see if the boundary condition has been applied correctly.

\subsubsection{Automated Unit Testing with GitHub Actions}
To make sure that changes to the code do not break functionality, I implemented automated testing using GitHub Actions.
I wrote a GitHub Action that runs the ``pytest'' unit tests on every pull request and commit to the repository.

\subsection{Simulation of a fluid flow around a cylinder}

To simulate the fluid flow around a cylinder I initialized a simulation environment of $6cm$ by $4cm$ with a $5mm$ cylinder. I set $\rho$ to be $0.6125 \frac{kg}{m^3}$ and $\nu$ to be $3 * 10^{-5} \frac{m^2}{s}$. These values are physically plausible for air. The left side boundary condition was set to a fixed velocity of $1 \frac{m}{s}$, and the top, right, and bottom boundaries were set as free slip conditions. The $dt$ was set to $0.15 \mu s$, and each step was a total of $30$ time steps, or $4.5 \mu s$. Every $4.5 \mu s$ I generated and saved the streamline plot for analysis. Three selected frames are shown in Figure \ref{fig:fluid_flow_cylinder_ns_fdm}.

The simulation was stable until $10$ time steps. After this point the simulation diverged. A longer simulation could be completed using a higher resolution grid, or by using smaller time steps.


\begin{figure}[!ht]
    \centering
    \begin{subfigure}{.3\textwidth}
        \centering
        \includegraphics[width=\linewidth]{Figures/cylinder_example_timesteps/streamline01.png}
        \caption{$9 \mu s$}
        \label{fig:sub1}
    \end{subfigure}%
    \begin{subfigure}{.3\textwidth}
        \centering
        \includegraphics[width=\linewidth]{Figures/cylinder_example_timesteps/streamline02.png}
        \caption{$13.5 \mu s$}
        \label{fig:sub2}
    \end{subfigure}
    \begin{subfigure}{.3\textwidth}
        \centering
        \includegraphics[width=\linewidth]{Figures/cylinder_example_timesteps/streamline09.png}
        \caption{$45 \mu s$}
        \label{fig:sub3}
    \end{subfigure}
    \caption{Three time steps of the FD simulation of fluid flow around a a cylinder. The pressure field is shown by the shading while streamlines are shown in purple.}
    \label{fig:fluid_flow_cylinder_ns_fdm}
\end{figure}

